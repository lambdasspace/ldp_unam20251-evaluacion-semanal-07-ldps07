Comenzamos construyendo el ambiente y se queda el cuerpo: \vspace{.3cm}

\textbf{Pila:}
\[
\begin{array}{c}
  a \Leftarrow 2
\end{array}
\]

\textbf{Nos queda:}
\begin{verbatim}
    (let (b 3)
        (let (foo (lambda (x) (- (+ a b) x)))
            (let (a -2)
                (let (b -3)
                    (let (foo (lambda (x) (+ (- a b) x)))
                        (foo - 10))))))
\end{verbatim}

Repetimos, metemos a b en la pila y seguimos con el cuerpo: \vspace{.3cm}

\textbf{Pila:}
\[
\begin{array}{c}
    b \Leftarrow 3 \\
    a \Leftarrow 2
\end{array}
\]

\textbf{Nos queda:}
\begin{verbatim}
    (let (foo (lambda (x) (- (+ a b) x)))
        (let (a -2)
            (let (b -3)
                (let (foo (lambda (x) (+ (- a b) x)))
                    (foo - 10)))))
\end{verbatim}

Repetimos, metemos foo en la pila y seguimos con el cuerpo (ojo que aqui
vamos a tener que crear una segunda pila que sera el ambiente de foo): \vspace{.3cm}

\textbf{Pila:}
\[
\begin{array}{c}
    foo \Leftarrow <x,(- (+ a b) x),\varepsilon_1> \\
    b \Leftarrow 3 \\
    a \Leftarrow 2
\end{array}
\]

\textbf{Pila $\varepsilon_1$:} (ambiente de foo)
\[
\begin{array}{c}
    b \Leftarrow 3 \\
    a \Leftarrow 2
\end{array}
\]

\textbf{Nos queda:}
\begin{verbatim}
    (let (a -2)
        (let (b -3)
            (let (foo (lambda (x) (+ (- a b) x)))
                (foo - 10))))
\end{verbatim}

Repetimos, metemos a en la pila y seguimos con el cuerpo: \vspace{.3cm}

\textbf{Pila:}
\[
\begin{array}{c}
    a \Leftarrow -2 \\
    foo \Leftarrow <x,(- (+ a b) x),\varepsilon_1> \\
    b \Leftarrow 3 \\
    a \Leftarrow 2
\end{array}
\]

\textbf{Pila $\varepsilon_1$:} (ambiente de foo)
\[
\begin{array}{c}
    b \Leftarrow 3 \\
    a \Leftarrow 2
\end{array}
\]

\textbf{Nos queda:}
\begin{verbatim}
    (let (b -3)
        (let (foo (lambda (x) (+ (- a b) x)))
            (foo - 10)))
\end{verbatim}

Repetimos, metemos b en la pila y seguimos con el cuerpo: \vspace{.3cm}

\textbf{Pila:}
\[
\begin{array}{c}
    b \Leftarrow -3 \\
    a \Leftarrow -2 \\
    foo \Leftarrow <x,(- (+ a b) x),\varepsilon_1> \\
    b \Leftarrow 3 \\
    a \Leftarrow 2
\end{array}
\]

\textbf{Pila $\varepsilon_1$:} (ambiente de foo)
\[
\begin{array}{c}
    b \Leftarrow 3 \\
    a \Leftarrow 2
\end{array}
\]

\textbf{Nos queda:}
\begin{verbatim}
    (let (foo (lambda (x) (+ (- a b) x)))
        (foo - 10))
\end{verbatim}

Repetimos, metemos foo en la pila y seguimos con el cuerpo (ojo que aqui
vamos a tener que crear una segunda pila que sera el ambiente de foo): \vspace{.3cm}

\textbf{Pila:}
\[
\begin{array}{c}
    foo \Leftarrow <x,(+ (- a b) x),\varepsilon_2> \\
    b \Leftarrow -3 \\
    a \Leftarrow -2 \\
    foo \Leftarrow <x,(- (+ a b) x),\varepsilon_1> \\
    b \Leftarrow 3 \\
    a \Leftarrow 2
\end{array}
\]

\textbf{Pila $\varepsilon_1$:} (ambiente de foo)
\[
\begin{array}{c}
    b \Leftarrow 3 \\
    a \Leftarrow 2
\end{array}
\]

\textbf{Pila $\varepsilon_2$:} (ambiente de foo)
\[
\begin{array}{c}
    b \Leftarrow -3 \\
    a \Leftarrow -2 \\
    foo \Leftarrow <x,(- (+ a b) x),\varepsilon_1> \\
    b \Leftarrow 3 \\
    a \Leftarrow 2
\end{array}
\]

\textbf{Nos queda:}
\begin{verbatim}
    (foo - 10)
\end{verbatim}

Ahora evaluamos, buscamos el valor de foo en la pila principal 
de arriba hacia abajo (agregamos el parametro con el valor a su ambiente): \vspace{.3cm}

\textbf{Nos queda:}
\begin{verbatim}
    (+ (- a b) x)
\end{verbatim}

\textbf{Pila $\varepsilon_2$:} (ambiente de foo)
\[
\begin{array}{c}
    x \Leftarrow -10 \\
    b \Leftarrow -3 \\
    a \Leftarrow -2 \\
    foo \Leftarrow <x,(- (+ a b) x),\varepsilon_1> \\
    b \Leftarrow 3 \\
    a \Leftarrow 2
\end{array}
\]

El resto de pilas se quedan igual. \vspace{.3cm}

Buscamos los valores de x,a y b en la pila $\varepsilon_2$: \vspace{.3cm}

\textbf{Nos queda:}
\begin{verbatim}
    (+ (- -2 -3) -10)
    (+ (1) -10)
    (-9)
\end{verbatim}

Resolviendo nos queda que la expresion es igual a -9.