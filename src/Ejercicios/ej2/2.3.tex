Comenzamos construyendo el ambiente y se queda el cuerpo: \vspace{.3cm}

\textbf{Pila:}
\[
\begin{array}{c}
  foo \Leftarrow <x,(+ x (foo (- x 1))),\varepsilon_1>
\end{array}
\]

\textbf{Pila $\varepsilon_1$:} (ambiente de foo)
\[
\begin{array}{c}
    \emptyset 
\end{array}
\]

\textbf{Nos queda:}
\begin{verbatim}
        (foo 10)
\end{verbatim}

Ahora, evaluamos foo con 10: \vspace{.3cm}

\textbf{Pila $\varepsilon_1$:} (ambiente de foo)
\[
\begin{array}{c}
    x \Leftarrow 10
\end{array}
\]

\textbf{Nos queda:}
\begin{verbatim}
        (+ x (foo (- x 1)))
\end{verbatim}

Sustituimos los valores y nos queda: \vspace{.3cm}
\begin{verbatim}
    (+ 10 (+ x (foo (- x 1))))
\end{verbatim}

\textbf{Pila $\varepsilon_1$:} (ambiente de foo)
\[
\begin{array}{c} 
    x \Leftarrow (- x 1) \\
    x \Leftarrow 10
\end{array}
\]

Y ahora como vemos tambien nos queda el mismo caso en donde recursivamente se llama a foo 
y se agrega un valor para x en el ambiente de foo, que al no haber caso base, se quedara en un bucle infinito.